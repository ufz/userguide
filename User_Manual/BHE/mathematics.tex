\chapter{Governing Equations and the Numerical Model}
Here in this chapter of the report, we summarize the governing equations of heat transport processes inside and around the Borehole Heat Exchangers (BHEs).  Also, the details of this mathematical formulation and the techniques applied to solve it numerically will be discussed. Note that the method implemented here is not the creative work of the authors, but rather a collection of contributions from the scientific community. The formulation of heat exchange between BHEs and the surrounding soil, was proposed by Al-Khoury et al \cite{AlKhoury2010}. This formulation was later-on adopted by Diersch et al. \cite{Diersch2011a} \cite{Diersch2011b} into the commercial software FEFLOW. In this work, we followed the same idea as Al-Khoury and Diersch, and implemented the heat transport process in response to BHEs into the open-source scientific software OpenGeoSys. For interested readers, the FEFLOW book \cite{FEFLOW2014} also serves a good reference for the better understanding of this work. 
%%
\begin{figure}
\begin{center}
\includegraphics[width=0.8\textwidth]{fig/crosssection_1U}
\end{center}
\caption{Configuration of the 1U type BHE and its corresponding resister-capacitor concept (Reproduced after the FEFLOW book \cite{FEFLOW2014})}
\label{fig:cross_sec_1u}
\end{figure}
%%
\section{Model Concept for BHEs}
To exchange heat with the surrounding soil and rock, borehole heat exchangers have different designs and configurations. The most popular designs can be categorized into 4 differnt types, i.e. the 1U, 2U, CXC and CXA. The U-type BHEs are named after the U shaped tubes laid vertically along the borehole. The number 1 or 2 refer to how many pairs of U tubes are in the same borehole. To illustrate this, Fig \ref{fig:cross_sec_1u} shows the horizontal cross-section of a 1U type BHE. Notice that the U tubes are normally sealed by grout materials, and not in direct contact with the soil. For a typical BHE, a refrigerant fluid is circulating inside of the U tube, absorbing or releasing heat into the surrounding grout and soil. 
%%
In order to numerically simulate the heat transfer inside of a BHE, the device is further conceptualized by the so-called Resister-Capacitor model. This idea originates from the electrical engineering field. In a electrical circuit, if the electric current is hindered by a component, we call this component a Resistor. When a component is capable of storing the electricity, we call it a Capacitor. So the same concept is also applied to the heat transport process in a BHE. Taking the 1U type of BHE in Figure X as an example, there are 4 temperature values assigned to a BHE cross-section at a particular depth. They are $T_s$, $T_{i1}$, $T_{o1}$, $T_{g1}$ and $T_{g2}$, referring to the temperatures of the surrounding soil, the inlet pipe, the outlet pie, the 1st and the 2nd grout zone respectively. As a convention, we always assume the 1st grout zone is the one surrounding the inlet pipe. Following this concept, the heat transfer between the pipe and the soil can be divided into five pathways: i) between inlet pipe and 1st grout zone; ii) between outlet pipe and 2nd grout zone; iii) between 2 grout zones; iv) between 1st grout zone and the soil; v) between 2nd grout zone and the soil. The heat flux $q_n$ on each of these pathways, are driven by the temperature difference and regulated by the heat transfer coefficient $\Phi$. For example, the heat flux from inlet pipe to the 1st grout zone can be calculated by 
\begin{equation}
\label{eqn:heat_transfer}
q_n = \Phi_{fig} \left( T_{i1} - T_{g1} \right), 
\end{equation} 
which is inversely dependent on the product of heat resistance $R$ and specific exchange area $S$
\begin{equation}
\Phi = \frac{1}{R S}
\end{equation}
Depending on the different pathways, we have corresponding heat transfer coefficients, denoted as $\Phi_{fig}$, $\Phi_{fog}$, $\Phi_{gg}$ and $\Phi_{gs}$. Details regarding how to calculate these heat transfer coefficients can be found in the paper of Diersch et al\cite{Diersch2011a}.  
~~\
\section{Governing Equations}
\subsection{Governing Equations for the Heat Transport Process in Soil}
For the heat transport process in the soil, the development of soil temperature $T_s$ is contributed by both the heat convection of the fluid $f$ in the soil and the heat conduction through the soil matrix. Let $\rho^s$, $\rho^f$ and $c^s$, $c^f$ be the density and specific heat capacity of fluid $f$ and soil $s$. If assuming the soil matrix is fully saturated with groundwater, the Darcy velocity of which is described by the vector $q$, then the conservation equation writes as
\begin{equation}
\label{eqn:heat_transport_soil}
\frac{\partial}{\partial t}  \left[ \epsilon \rho^f c^f + ( 1-\epsilon ) \rho^s c^s \right]  T_s 
+ \nabla \cdot \left(  \rho^f c^f \mathbf{q} T_s  \right) 
- \nabla \cdot \left(  \Lambda^s \cdot \nabla T_s  \right) = H_s,  
\end{equation}
with $\Lambda^s$ the tensor of thermal hydrodynamic dispersion and $H_s$ the source and sink terms for heat. When considering the heat exchange between the BHEs and the soil, the above governing equation is subject to a Cauchy-type of boundary condition: 
\begin{equation}
- \left(  \Lambda^s \cdot \nabla T_s  \right) = q_{n T_s}. 
\end{equation}
\subsection{Governing Equations for the Borehole Heat Exchangers}
The governing equations for the BHE write differently, depending on whether it is for the pipelines or for the grout zones. For the pipelines, the heat transport process is dominated by the convection of the refrigerant. 
\begin{eqnarray}
\label{eqn:gov_eqn_bhe_pipe}
\rho^r c^r \frac{\partial T_k}{\partial t} 
+ \rho^r c^r \mathbf{u} \cdot \nabla T_k 
- \nabla \cdot \left( \Lambda^r \cdot \nabla T_k \right)
&=& H_k \mathrm{~in~} \Omega_k \nonumber \\
\mathrm{with~Cauchy~type~of~BC:~} -\left( \Lambda^r \cdot \nabla T_k \right) \cdot n 
&=& q_{nT_k} \mathrm{~on~} \Gamma_k   \nonumber \\
\mathrm{for~} k &=& i1, o1, (i2, o2) 
\end{eqnarray}
$\Lambda^r$ stands for the hydrodynamic thermodispersion for the refrigerant,  
\begin{equation}
\Lambda^r= \left( \lambda^r + \rho^r c^r \beta_L \| \bm{u} \| \right) \delta. 
\end{equation}
For the grout zones, the heat transport is mainly controlled by the heat dissipation. 
\begin{eqnarray}
\label{eqn:gov_eqn_bhe_grout}
\varepsilon^g \rho^g c^g \frac{\partial T_k}{\partial t} 
- \nabla \cdot \left( \varepsilon^g \lambda^g \cdot \nabla T_k \right)
&=& H_k \mathrm{~in~} \Omega_k \nonumber \\
\mathrm{with~Cauchy~type~of~BC:~} -\left( \varepsilon^g \lambda^g \cdot \nabla T_k \right) \cdot n 
&=& q_{nT_k} \mathrm{~on~} \Gamma_k   \nonumber \\
\mathrm{for~} k &=& g1, (g2, (g3, g4)) 
\end{eqnarray}

\subsection{Calculation of the Cauchy type of Boundary Conditions}
The Cauchy type of boundary conditions exit for the soil part, for the pipelines, and also for the grout zones. They are regulating the heat exchange terms between these zones. In general, the heat exchange flux $q_{nT_k}$ is proportional to the difference of the Temperatures (see Equation \ref{eqn:heat_transfer}). The calculation of these flux terms is summarized in Table \ref{tab:heat_ex_terms}. 

\begin{table}
\centering
\caption{Boundary heat fluxes $q_{nTk}$ for different types of BHEs, reproduced after the FEFLOW book \cite{FEFLOW2014}. }
\label{tab:heat_ex_terms}
\begin{tabular}{p{1cm}p{3cm}p{3cm}p{3cm}p{3cm}}
\hline \hline
k              & 2U & 1U & CXA & CXC \\
\hline \hline
\multirow{2}{*}{i1}    & $-\Phi_{fig}^{2U}(T_{g1}-T_{i1})$ 
                       & $-\Phi_{fig}^{1U}(T_{g1}-T_{i1})$  
                       & $-\Phi_{fig}^{CXA}(T_{g1}-T_{i1})$
                       & $-\Phi_{ff}^{CXC}(T_{o1}-T_{i1})$ \\
                       & & & $-\Phi_{ff}^{CXA}(T_{o1}-T_{i1})$ & \\
\hline
i2             & $-\Phi_{fig}^{2U}(T_{g2}-T_{i2})$
               & -  & -  & - \\
\hline
\multirow{2}{*}{o1}    & $-\Phi_{fog}^{2U}(T_{g3}-T_{o1})$
                       & $-\Phi_{fog}^{1U}(T_{g2}-T_{o1})$  
                       & $-\Phi_{ff}^{CXA}(T_{i1}-T_{o1})$               
                       & $-\Phi_{fog}^{CXC}(T_{g1}-T_{o1})$ \\
                       & & & & $-\Phi_{ff}^{CXC}(T_{i1}-T_{o1})$\\
\hline
o2             & $-\Phi_{fog}^{2U}(T_{g4}-T_{o2})$
               & -  & -  & - \\
\hline
\multirow{5}{*}{g1}    & $-\Phi_{gs}^{2U}(T_{s}-T_{g1})  $ & $-\Phi_{gs}^{1U}(T_{s}-T_{g1})$     
               & $-\Phi_{gs}^{CXC}(T_{s}-T_{g1})$ & $-\Phi_{gs}^{CXA}(T_{s}-T_{g1})$ \\   
                       & $-\Phi_{fig}^{2U}(T_{i1}-T_{g1})$ & $-\Phi_{fig}^{1U}(T_{i1}-T_{g1})$  
               & $-\Phi_{fog}^{CXC}(T_{o1}-T_{g1})$ & $-\Phi_{gs}^{CXA}(T_{s}-T_{g1})$ \\
                       & $-\Phi_{gg2}^{2U}(T_{g2}-T_{g1})$ & $-\Phi_{gg}^{1U}(T_{g2}-T_{g1})$    &-&-\\
                       & $-\Phi_{gg1}^{2U}(T_{g3}-T_{g1})$ &                                     &~&~\\
                       & $-\Phi_{gg1}^{2U}(T_{g4}-T_{g1})$ &                                     &~&~\\
\hline
\multirow{5}{*}{g2}    & $-\Phi_{gs}^{2U}(T_{s}-T_{g2})$   & $-\Phi_{gs}^{1U}(T_{s}-T_{g2})$     &~&~\\ 
                       & $-\Phi_{fig}^{2U}(T_{i2}-T_{g2})$ & $-\Phi_{fog}^{1U}(T_{o1}-T_{g2})$   &~&~\\
                       & $-\Phi_{gg2}^{2U}(T_{g1}-T_{g2})$ & $-\Phi_{gg}^{1U}(T_{g1}-T_{g2})$    &-&-\\
                       & $-\Phi_{gg1}^{2U}(T_{g3}-T_{g2})$ &                                     &~&~\\
                       & $-\Phi_{gg1}^{2U}(T_{g4}-T_{g2})$ &                                     &~&~\\
\hline
\multirow{5}{*}{g3}    & $-\Phi_{gs}^{2U}(T_{s}-T_{g3})$   &                                     &~&~\\
                       & $-\Phi_{fig}^{2U}(T_{o1}-T_{g3})$ &                                     &~&~\\
                       & $ -\Phi_{gg2}^{2U}(T_{g4}-T_{g3})$&-                                    &-&-\\
                       & $-\Phi_{gg1}^{2U}(T_{g1}-T_{g3})$ &                                     &~&~\\
                       & $-\Phi_{gg1}^{2U}(T_{g2}-T_{g3})$ &                                     &~&~\\
\hline
\multirow{5}{*}{g4}    & $-\Phi_{gs}^{2U}(T_{s}-T_{g4})$   &                                     &~&~\\
                       & $-\Phi_{fig}^{2U}(T_{o2}-T_{g4})$ &                                     &~&~\\
                       & $-\Phi_{gg2}^{2U}(T_{g3}-T_{g4})$ &-                                    &-&-\\
                       & $-\Phi_{gg1}^{2U}(T_{g1}-T_{g4})$ &                                     &~&~\\
                       & $-\Phi_{gg1}^{2U}(T_{g2}-T_{g4})$ &                                     &~&~\\
\hline \hline
\end{tabular}
\end{table}

\section{Numerical Model}
%%
\begin{figure}
\begin{center}
\includegraphics[width=0.8\textwidth]{fig/simple_mesh_geometry}
\end{center}
\caption{Mesh and geometry structure, with soil domain represented by prism elements and BHE by line elements. }
\label{fig:simple_mesh}
\end{figure}
%%

\subsection{Mesh Arrangement}
To simulate the heat transport process together with BHEs, a dual-continuum approach has been adopted to treat the soil and BHEs part respectively. For the soil part, prism elements are used to discretize the 3D domain. In addition to that, 1D line elements along the edge of the prism elements are chosen to form the 2nd domain, representing BHEs. The example illustrated in Figure \ref{fig:simple_mesh} contains a mesh structure with 12 nodes and 6 element. Node "0" - "11" forms the 3 prism elements, which refers to the the soil domain. The BHE domain is composed of 3 line elements in red color. Notice the node "2", "5", "8" and "11" are both employed by the prim and the line elements. As a result, the total number of nodes remains the same and the number of elements slightly increases. In soil domain, there is only one primary variable, the soil temperature, on each node. While in the BHE domain, each node has 4, 8, or 3 primary variables, depending on the type of the BHE. Table \ref{tab:bhe_primary_variables} summarizes these combinations. 
%%
\begin{table}
\caption{Different combination of primary variables for the borehole heat exchangers. }
\label{tab:bhe_primary_variables}
\centering
\begin{tabular}{l l l}
\hline
Type of BHE    & Number of Primary Variables & Combination of Primary Variables \\
\hline
1U             & 4 & $\mathrm{T_{i1},~T_{o1},~T_{g1},~T_{g2}}$\\
2U             & 8 & $\mathrm{T_{i1},~T_{i2},~T_{o1},~T_{o2},~T_{g1},~T_{g2},~T_{g3},~T_{g4}}$\\
CXA            & 3 & $\mathrm{T_{i1},~T_{o1},~T_{g1}}$\\
CXC            & 3 & $\mathrm{T_{i1},~T_{o1},~T_{g1}}$ \\
\hline
\end{tabular}
\end{table}

The following text box shows the content of the mesh file for OpenGeoSys. Under the key word \textit{\$NODES}, the number \textit{12} tells the software there are all together 12 nodes. This is followed by a section of node coordinate information. Each node begins with a index number, then the x, y, and z coordinates. The keyword \$ELEMENTS signals the begin of element information. Here we have 6 elements, with 3 prisms and 3 lines. Similar to the node information, each element begins with an index value, then a second index linking to the corresponding material properties. The keywords \textit{pris} and \textit{line} reflects the type of the element, followed by the index of nodes that are connected to this element. 
%%
\begin{Verbatim}[gobble=0, 
                 frame=single, 
                 label=Mesh File, 
                 numbers=left]
#FEM_MSH
 $PCS_TYPE
  GROUNDWATER_FLOW
 $NODES
  12 
0 0 0 0
1 0 1 0
2 1 0 0
3 0 1 1
4 0 0 1
5 1 0 1
6 0 1 2
7 0 0 2
8 1 0 2
9 0 1 3
10 0 0 3
11 1 0 3
  $ELEMENTS
  6 
0 0 pris 2 1 0 5 3 4
1 0 pris 5 3 4 8 6 7
2 0 pris 8 6 7 11 9 10 
3 1 line 9 6
4 1 line 6 3
5 1 line 3 1
#STOP
\end{Verbatim}

\subsection{Finite Element Discretization}
For the domain of BHEs, the governing equations \ref{eqn:gov_eqn_bhe_pipe} and \ref{eqn:gov_eqn_bhe_grout} are discretized by finite elements. If introducing the spatial weighting function $\omega$, the weak statements wirte as
\begin{eqnarray}
\label{eqn:gov_eqn_bhe_pipe_dis}
\int_{\Omega_k} \left[ \omega \rho^r c^r \left( \frac{\partial T_k}{\partial t} + \bm{u} \cdot \nabla T_k \right) + \nabla \omega \cdot \left( \Lambda^r \cdot \nabla T_k \right) \right] d\Omega &=&~ \nonumber\\
-\int_{\Gamma_k} \omega q_{nT_k} d\Gamma_k + \int_{\Omega_k} \omega H_k d\Omega &~& \mathrm{~for~k~}=~i1,~o1,~(i2,~o2)
\end{eqnarray}
for the pipelines, and 
\begin{eqnarray}
\label{eqn:gov_eqn_bhe_grout_dis}
\int_{\Omega_k} \left[ \omega \rho^g c^g \frac{\partial T_k}{\partial t} + \nabla \omega \cdot \left( \Lambda^g \cdot \nabla T_k \right) \right] d\Omega &=&~ \nonumber\\
-\int_{\Gamma_k} \omega q_{nT_k} d\Gamma_k + \int_{\Omega_k} \omega H_k d\Omega &~& \mathrm{~for~k~}=~g1,~(g2,~(g3,~g4))
\end{eqnarray}
for the grout zones. If using Galerkin Finite Element Method (GFEM), the above equation \ref{eqn:gov_eqn_bhe_pipe_dis} and \ref{eqn:gov_eqn_bhe_grout_dis} can be written in the matrix form. 
\begin{equation}
\label{eqn:gov_eqn_bhe_matrix}
\bm{P}^\pi \cdot \bm{\dot{T}}^\pi + ( \bm{L^\pi} + \bm{R^\pi} ) \cdot \bm{T^\pi} = \bm{W}^\pi - \bm{R}^{\pi s} \cdot \bm{T}^{s}
\end{equation}
with the time derivative related mass matrix $\bm{P}^\pi$ formulated as
\begin{equation}
\bm{P}^\pi = \left\{ \begin{array}{ll}
\left( \begin{array}{cccccccc}
\bm{P}_{i1} & 0 & 0 & 0 & 0 & 0 & 0 & 0 \\
0 & \bm{P}_{i2} & 0 & 0 & 0 & 0 & 0 & 0 \\
0 & 0 & \bm{P}_{o1} & 0 & 0 & 0 & 0 & 0 \\
0 & 0 & 0 & \bm{P}_{o2} & 0 & 0 & 0 & 0 \\
0 & 0 & 0 & 0 & \bm{P}_{g1} & 0 & 0 & 0 \\
0 & 0 & 0 & 0 & 0 & \bm{P}_{g2} & 0 & 0 \\
0 & 0 & 0 & 0 & 0 & 0 & \bm{P}_{g3} & 0 \\
0 & 0 & 0 & 0 & 0 & 0 & 0 & \bm{P}_{g4} \end{array} \right)
  & \mbox{2U} \\
\left( \begin{array}{cccc}
\bm{P}_{i1} & 0 & 0 & 0  \\
0 & \bm{P}_{o1} & 0 & 0  \\
0 & 0 & \bm{P}_{g1} & 0  \\
0 & 0 & 0 & \bm{P}_{g2}  \end{array} \right)
 & \mbox{1U} \\
\left( \begin{array}{ccc}
\bm{P}_{i1} & 0 & 0  \\
0 & \bm{P}_{o1} & 0  \\
0 & 0 & \bm{P}_{g1}  \end{array} \right)
 & \mbox{CXA or CXC. } \\
       \end{array} \right.
\end{equation}
For the pipeline or grout zone of the BHE, the mass matrix writes
\begin{equation}
\bm{P}_k = \left\{ \begin{array}{rcl}
\Sigma_{e} \int_{\Omega_k^e} \rho^r c^r N_i N_j d\Omega^e 
   &\mbox{~for~}& k = i1, o1, (i2, o2) \\
\Sigma_{e} \int_{\Omega_k^e} \rho^g c^g N_i N_j d\Omega^e 
   &\mbox{~for~}& k = g1, (g2, (g3, g4)). 
       \end{array} \right.
\end{equation}
The heat exchange terms are summarized in the matrix $\bm{R}^\pi$, 
\begin{equation}
\label{eqn:heat_exchange_term_R}
\bm{R}^\pi = \left\{ \begin{array}{ll}
\scalemath{0.5}{
\left( \begin{array}{cccccccc}
\bm{R}_{i1} + \bm{R}_{io} & 0 & -\bm{R}_{io} & 0 & -\bm{R}_{i1} & 0 & 0 & 0 \\
0 & \bm{R}_{i2} & 0 & 0 & 0 & -\bm{R}_{i2} & 0 & 0 \\
-\bm{R}_{io} & 0 & \bm{R}_{io} + \bm{R}_{o1} & 0 & 0 & 0 & -\bm{R}_{o1} & 0 \\
0 & 0 & 0 & \bm{R}_{o2} & 0 & 0 & 0 & -\bm{R}_{o2} \\
-\bm{R}_{i1} & 0 & 0 & 0 & \bm{R}_{i1}+2\bm{R}_{g1}+\bm{R}_{g2}+\bm{R}_{s} & -\bm{R}_{g2} & -\bm{R}_{g1} & -\bm{R}_{g1} \\
0 & -\bm{R}_{i2} & 0 & 0 & -\bm{R}_{g2} & \bm{R}_{i2}+2\bm{R}_{g1}+\bm{R}_{g2}+\bm{R}_{s} & -\bm{R}_{g1} & -\bm{R}_{g1} \\
0 & 0 & -\bm{R}_{o1} & 0 & -\bm{R}_{g1} & -\bm{R}_{g1} & \bm{R}_{o1}+2\bm{R}_{g1}+\bm{R}_{g2}+\bm{R}_{s} & 0 \\
0 & 0 & 0 & -\bm{R}_{o2} & -\bm{R}_{g1} & -\bm{R}_{g1} & -\bm{R}_{g2} & \bm{R}_{o2}+2\bm{R}_{g1}+\bm{R}_{g2}+\bm{R}_{s} \end{array} \right)
}
  & \mbox{2U} \\
\scalemath{0.7}{
\left( \begin{array}{cccc}
\bm{R}_{i1}+\bm{R}_{io} & -\bm{R}_{io} & -\bm{R}_{i1} & 0  \\
-\bm{R}_{io} & \bm{R}_{o1}+\bm{R}_{io} & 0 & -\bm{R}_{o1}  \\
-\bm{R}_{i1} & 0 & \bm{R}_{i1}+\bm{R}_{g1} & -\bm{R}_{g1}  \\
0 & -\bm{R}_{o1} & -\bm{R}_{g1} & \bm{R}_{o1}+\bm{R}_{g1}  \end{array} \right)
}
 & \mbox{1U} \\
\scalemath{0.7}{
\left( \begin{array}{ccc}
\bm{R}_{i1}+\bm{R}_{io} & -\bm{R}_{io} & -\bm{R}_{i1}  \\
-\bm{R}_{io} & \bm{R}_{io} & 0  \\
-\bm{R}_{i1} & 0 & \bm{R}_{i1}  \end{array} \right)
}
 & \mbox{CXA} \\
\scalemath{0.7}{
\left( \begin{array}{ccc}
\bm{R}_{io} & -\bm{R}_{io} & 0  \\
-\bm{R}_{io} & \bm{R}_{io}+\bm{R}_{o1} & -\bm{R}_{o1}  \\
0 & -\bm{R}_{o1} & \bm{R}_{o1}  \end{array} \right)
}
 & \mbox{CXA} 
       \end{array} \right.
\end{equation}
\begin{equation}
\bm{L}^\pi = \left\{ \begin{array}{ll}
\left( \begin{array}{cccccccc}
\bm{L}_{i1} & 0 & 0 & 0 & 0 & 0 & 0 & 0 \\
0 & \bm{L}_{i2} & 0 & 0 & 0 & 0 & 0 & 0 \\
0 & 0 & \bm{L}_{o1} & 0 & 0 & 0 & 0 & 0 \\
0 & 0 & 0 & \bm{L}_{o2} & 0 & 0 & 0 & 0 \\
0 & 0 & 0 & 0 & \bm{L}_{ig} & 0 & 0 & 0 \\
0 & 0 & 0 & 0 & 0 & \bm{L}_{ig} & 0 & 0 \\
0 & 0 & 0 & 0 & 0 & 0 & \bm{L}_{og} & 0 \\
0 & 0 & 0 & 0 & 0 & 0 & 0 & \bm{L}_{og} \end{array} \right)
  & \mbox{2U} \\
\left( \begin{array}{cccc}
\bm{L}_{i1} & 0 & 0 & 0  \\
0 & \bm{L}_{o1} & 0 & 0  \\
0 & 0 & \bm{L}_{g1} & 0  \\
0 & 0 & 0 & \bm{L}_{g2}  \end{array} \right)
 & \mbox{1U} \\
\left( \begin{array}{ccc}
\bm{L}_{i1} & 0 & 0  \\
0 & \bm{L}_{o1} & 0  \\
0 & 0 & \bm{L}_{g1}  \end{array} \right)
 & \mbox{CXA or CXC} \\
       \end{array} \right.
\end{equation}
For the pipeline part, the transport operator $L$ is composed of both advection and dispersion terms, 
\begin{equation}
\begin{array}{lll}
\bm{L}_k = \Sigma_{e} \int_{\Omega_k^e} \left( N_i \rho^r c^r \nabla N_j 
+ \nabla N_i \cdot ( \Lambda \cdot \nabla N_j ) \right) d\Omega^e 
&\mbox{~for~}& k = i1, o1, (i2, o2). 
\end{array}
\end{equation}
While for the grout zones, only the heat dissipation terms contribute, 
\begin{equation}
\begin{array}{lll}
\bm{L}_{ig} = \bm{L}_{og} = \Sigma_{e} \int_{\Omega_k^e} \nabla N_i \cdot ( \Lambda^g \cdot \nabla N_j ) d\Omega^e 
&\mbox{~for~}& k = (g1, g2, g3, g4). 
\end{array}
\end{equation}
For the source and sink terms, 
\begin{equation}
\begin{array}{lll}
\bm{W}_{k} = \Sigma_{e} \int_{\Omega_k^e} N_i H_k d\Omega^e 
&\mbox{~for~}& \forall k. 
\end{array}
\end{equation}
For the heat exchange terms $\bm{R_{\pi s}}$ and $\bm{R_{s \pi}}$, 
\begin{equation}
\bm{R}^{s \pi} = {\bm{R}^{\pi s}}^{T} = \left\{ \begin{array}{ll}
(\bm{0}~\bm{0}~\bm{0}~\bm{0}~\bm{-R_s}~\bm{-R_s}~\bm{-R_s}~\bm{-R_s}) 
   & \mbox{~~2U} \\
(\bm{0}~\bm{0}~\bm{-R_s}~\bm{-R_s}) 
   & \mbox{~~1U} \\
(\bm{0}~\bm{0}~\bm{-R_s}) 
   & \mbox{~~CXA or CXC} 
       \end{array} \right.
\end{equation}
\subsection{Assembly of the Global Equation System}
In order to simulate the heat transfer between BHEs and the surrounding soil, the governing equations for the pipeline and grout zones must be assemble into a global matrix system together with the linearized heat transport equaiton of the soil. When the equation \ref{eqn:gov_eqn_bhe_matrix} is combined with the matrix form of equation \ref{eqn:heat_transport_soil}, the global matrix system writes as
\begin{equation}
\label{eqn:global_matrix_form}
\left( \begin{array}{cc}
\bm{P}^s & \bm{0} \\
\bm{0}   & \bm{P}^\pi
\end{array} \right) 
\cdot 
\left(
\begin{array}{c}
\bm{\dot{T}}^s \\
\bm{\dot{T}}^\pi
\end{array} \right) 
+ \left( \begin{array}{cc}
\bm{L}^s - \bm{R}^\pi & \bm{R}^{s \pi}\\
\bm{R}^{\pi s} & \bm{T}^\pi
\end{array}
\right) 
\cdot 
\left(\begin{array}{c}
\bm{T}^s \\
\bm{T}^\pi
\end{array} \right) 
= \left( 
\begin{array}{c}
\bm{W}^s \\
\bm{W}^\pi \\
\end{array} \right). 
\end{equation}

When Euler time discretization is applied on the above equation, the fully linearized global matrix system looks like
\begin{equation}
\left(
\begin{array}{cc}
\bm{A}^s & \bm{R}^{s \pi} \\
\bm{R}^{s \pi} & \bm{A}^\pi
\end{array}
\right) 
\cdot 
\left(
\begin{array}{c}
\bm{T}^s \\
\bm{T}^\pi
\end{array}
\right)_{n+1}
= 
\left( 
\begin{array}{c}
\bm{B}^s \\
\bm{B}^\pi
\end{array}
\right)_{n+1, n} , 
\end{equation}
where $n$ and $n+1$ represents the previous and current time step. If a corrector recurrence scheme is applied, then the left-hand-side matrix and right-hand-side vectors writes as 
\begin{eqnarray}
\label{eqn:linearized_matrix_system}
\bm{A}^s & = & \frac{1}{\Delta t_n} \bm{P}^s + \theta \left( \bm{L}^s - \bm{R}^\pi \right) \nonumber \\ 
\bm{B}^s & = & \left( \frac{1}{\Delta t_n} \bm{P}^s - (1 - \theta) \left( \bm{L}^s - \bm{R}^\pi \right) \right) \cdot \bm{T}_n^s + \bm{W}_{n+1}^s \theta + \bm{W}_{n}^s (1 - \theta) \nonumber \\
\bm{A}^\pi & = & \frac{1}{\Delta t_n} \bm{P}^\pi + \theta \bm{L}^\pi \nonumber \\ 
\bm{B}^\pi & = & \left( \frac{1}{\Delta t_n} \bm{P}^\pi - (1 - \theta) \bm{L}^\pi \right) \cdot \bm{T}_n^\pi + \bm{W}_{n+1}^\pi \theta + \bm{W}_{n}^\pi (1 - \theta) 
\end{eqnarray}
\subsection{Picard Iterations and Time Step Schemes}
After the $\bm{A}$ matrices and $\bm{B}$ vectors have been assembled, the OpenGeoSys software employs a linear solver the calculate the $\bm{T}$ values for the new time step, based on the previous time step value. Notice that, the heat exchange coefficients $\bm{R^{s \pi}}$ and $\bm{R^{\pi s}}$ will be multiplied with the temperature values, and producing the heat exchange flux between the soil and the BHE domain. This term is linearly dependent on the temperature difference. However, when a new set of temperature values are produced, the heat flux changes respectively. Therefore, a Picard iteration scheme has been employed by the OpenGeoSys software, to solve for converged temperature values. The following log message from a typical simulation run demonstrates the Picard iteration behavior. 
\begin{Verbatim}[gobble=0, 
                 frame=single, 
                 label=Log File of A Simulation Run, 
                 numbers=left]
      ================================================
    ->Process 1: HEAT_TRANSPORT_BHE
      ================================================
    PCS non-linear iteration: 0/2
      Assembling equation system...
      Calling linear solver...
	  SpBICGSTAB iteration: 2/1000 
      -->End of PICARD iteration: 0/2
         PCS error: 0.00110317
         ->Euclidian norm of unknowns: 0.00596578
    PCS non-linear iteration: 1/2
      Assembling equation system...
      Calling linear solver...
	  SpBICGSTAB iteration: 2/1000 
      -->End of PICARD iteration: 1/2
         PCS error: 1.37569e-006
         ->Euclidian norm of unknowns: 1.82299e-005
This step is accepted.
Data output: Polyline profile - BHE_1
#############################################################
\end{Verbatim}
As can be found in the log file, one iteration of the HEAT\_TRANSPORT\_BHE process is needed. Conventionally, the tolerance of Picard iteration is set to $1.0 \times 10^{-4}$, and the simulation normally converges after one iteration. 

In the work of Diersch et al \cite{Diersch2011a} \cite{FEFLOW2014}, they suggested to inverse the matrix system for BHE domain separately, then integrate its influence into the soil domain by a \textit{Schur complement} operation. Such procedures is reasonable as the BHE domain is typically small (degree of freedom number in the order of thousands by thousands). However, non-linearity of the governing equations cannot be eliminated without an iteration step. Their test run also shows that one iteration is necessary to obtain the accurate temperature values. Considering only little effort is added to the linear solver, we choose to directly iterate the linearized matrix system of \ref{eqn:linearized_matrix_system}. Our simulation shows that convergence can be achieve after one Picard iteration in nearly all simulations. 

% TODO, time stepping scheme. 


